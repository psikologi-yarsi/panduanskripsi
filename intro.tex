% Options for packages loaded elsewhere
% Options for packages loaded elsewhere
\PassOptionsToPackage{unicode}{hyperref}
\PassOptionsToPackage{hyphens}{url}
\PassOptionsToPackage{dvipsnames,svgnames,x11names}{xcolor}
%
\documentclass[
  letterpaper,
  DIV=11,
  numbers=noendperiod]{scrreprt}
\usepackage{xcolor}
\usepackage[margin=1in]{geometry}
\usepackage{amsmath,amssymb}
\setcounter{secnumdepth}{-\maxdimen} % remove section numbering
\usepackage{iftex}
\ifPDFTeX
  \usepackage[T1]{fontenc}
  \usepackage[utf8]{inputenc}
  \usepackage{textcomp} % provide euro and other symbols
\else % if luatex or xetex
  \usepackage{unicode-math} % this also loads fontspec
  \defaultfontfeatures{Scale=MatchLowercase}
  \defaultfontfeatures[\rmfamily]{Ligatures=TeX,Scale=1}
\fi
\usepackage{lmodern}
\ifPDFTeX\else
  % xetex/luatex font selection
\fi
% Use upquote if available, for straight quotes in verbatim environments
\IfFileExists{upquote.sty}{\usepackage{upquote}}{}
\IfFileExists{microtype.sty}{% use microtype if available
  \usepackage[]{microtype}
  \UseMicrotypeSet[protrusion]{basicmath} % disable protrusion for tt fonts
}{}
\makeatletter
\@ifundefined{KOMAClassName}{% if non-KOMA class
  \IfFileExists{parskip.sty}{%
    \usepackage{parskip}
  }{% else
    \setlength{\parindent}{0pt}
    \setlength{\parskip}{6pt plus 2pt minus 1pt}}
}{% if KOMA class
  \KOMAoptions{parskip=half}}
\makeatother
% Make \paragraph and \subparagraph free-standing
\makeatletter
\ifx\paragraph\undefined\else
  \let\oldparagraph\paragraph
  \renewcommand{\paragraph}{
    \@ifstar
      \xxxParagraphStar
      \xxxParagraphNoStar
  }
  \newcommand{\xxxParagraphStar}[1]{\oldparagraph*{#1}\mbox{}}
  \newcommand{\xxxParagraphNoStar}[1]{\oldparagraph{#1}\mbox{}}
\fi
\ifx\subparagraph\undefined\else
  \let\oldsubparagraph\subparagraph
  \renewcommand{\subparagraph}{
    \@ifstar
      \xxxSubParagraphStar
      \xxxSubParagraphNoStar
  }
  \newcommand{\xxxSubParagraphStar}[1]{\oldsubparagraph*{#1}\mbox{}}
  \newcommand{\xxxSubParagraphNoStar}[1]{\oldsubparagraph{#1}\mbox{}}
\fi
\makeatother


\usepackage{longtable,booktabs,array}
\usepackage{calc} % for calculating minipage widths
% Correct order of tables after \paragraph or \subparagraph
\usepackage{etoolbox}
\makeatletter
\patchcmd\longtable{\par}{\if@noskipsec\mbox{}\fi\par}{}{}
\makeatother
% Allow footnotes in longtable head/foot
\IfFileExists{footnotehyper.sty}{\usepackage{footnotehyper}}{\usepackage{footnote}}
\makesavenoteenv{longtable}
\usepackage{graphicx}
\makeatletter
\newsavebox\pandoc@box
\newcommand*\pandocbounded[1]{% scales image to fit in text height/width
  \sbox\pandoc@box{#1}%
  \Gscale@div\@tempa{\textheight}{\dimexpr\ht\pandoc@box+\dp\pandoc@box\relax}%
  \Gscale@div\@tempb{\linewidth}{\wd\pandoc@box}%
  \ifdim\@tempb\p@<\@tempa\p@\let\@tempa\@tempb\fi% select the smaller of both
  \ifdim\@tempa\p@<\p@\scalebox{\@tempa}{\usebox\pandoc@box}%
  \else\usebox{\pandoc@box}%
  \fi%
}
% Set default figure placement to htbp
\def\fps@figure{htbp}
\makeatother





\setlength{\emergencystretch}{3em} % prevent overfull lines

\providecommand{\tightlist}{%
  \setlength{\itemsep}{0pt}\setlength{\parskip}{0pt}}



 


% Warna judul agar mirip contoh (beige keemasan halus)
\usepackage{xcolor}
\definecolor{titleaccent}{RGB}{183,176,138} % #b7b08a
\KOMAoption{captions}{tableheading}
\makeatletter
\@ifpackageloaded{caption}{}{\usepackage{caption}}
\AtBeginDocument{%
\ifdefined\contentsname
  \renewcommand*\contentsname{Table of contents}
\else
  \newcommand\contentsname{Table of contents}
\fi
\ifdefined\listfigurename
  \renewcommand*\listfigurename{List of Figures}
\else
  \newcommand\listfigurename{List of Figures}
\fi
\ifdefined\listtablename
  \renewcommand*\listtablename{List of Tables}
\else
  \newcommand\listtablename{List of Tables}
\fi
\ifdefined\figurename
  \renewcommand*\figurename{Figure}
\else
  \newcommand\figurename{Figure}
\fi
\ifdefined\tablename
  \renewcommand*\tablename{Table}
\else
  \newcommand\tablename{Table}
\fi
}
\@ifpackageloaded{float}{}{\usepackage{float}}
\floatstyle{ruled}
\@ifundefined{c@chapter}{\newfloat{codelisting}{h}{lop}}{\newfloat{codelisting}{h}{lop}[chapter]}
\floatname{codelisting}{Listing}
\newcommand*\listoflistings{\listof{codelisting}{List of Listings}}
\makeatother
\makeatletter
\makeatother
\makeatletter
\@ifpackageloaded{caption}{}{\usepackage{caption}}
\@ifpackageloaded{subcaption}{}{\usepackage{subcaption}}
\makeatother
\usepackage{bookmark}
\IfFileExists{xurl.sty}{\usepackage{xurl}}{} % add URL line breaks if available
\urlstyle{same}
\hypersetup{
  colorlinks=true,
  linkcolor={blue},
  filecolor={Maroon},
  citecolor={Blue},
  urlcolor={Blue},
  pdfcreator={LaTeX via pandoc}}


\author{}
\date{}
\begin{document}


\chapter*{Pendahuluan}\label{pendahuluan}
\addcontentsline{toc}{chapter}{Pendahuluan}

Dalam perjalanan akademis, penulisan skripsi merupakan salah satu fase
krusial yang harus dilalui oleh setiap mahasiswa di perguruan tinggi.
Skripsi tidak hanya berfungsi sebagai salah satu syarat untuk mencapai
gelar akademik, tetapi juga sebagai bukti kemampuan mahasiswa dalam
melakukan penelitian yang sistematis dan mendalam terhadap suatu
masalah. Melalui skripsi, mahasiswa ditantang untuk menerapkan
pengetahuan yang telah mereka peroleh selama perkuliahan, sekaligus
mengembangkan kemampuan analisis, kritis, dan ilmiah mereka.

Tujuan dari buku ini adalah untuk memberikan panduan praktis dan teknis
dalam menulis skripsi. Buku ini dirancang untuk membantu mahasiswa dalam
merencanakan, melakukan dan menyajikan hasil penelitian mereka dalam
bentuk skripsi yang sistematis dan logis. Melalui panduan ini, mahasiswa
diharapkan dapat memahami langkah-langkah penulisan skripsi, mulai dari
pemilihan topik yang relevan, penyusunan proposal penelitian,
pengumpulan dan analisis data, hingga penyajian hasil penelitian dan
kesimpulan.

Penggunaan buku ini disarankan untuk dijadikan sebagai acuan atau
pedoman selama proses penulisan skripsi. Buku ini disusun dengan bahasa
yang mudah dipahami dan disertai dengan contoh serta tips praktis yang
akan sangat membantu dalam memecahkan berbagai kendala yang sering
dihadapi mahasiswa selama proses penulisan skripsi. Di awal buku,
pembaca akan dibawa untuk memahami konsep-konsep dasar skripsi dan
perbedaannya dengan karya ilmiah lainnya. Kemudian, dilanjutkan dengan
pembahasan secara mendetil mengenai posisi skripsi dalam kurikulum
Fakultas Psikologi Universitas YARSI. Pembahasannya meliputi persyaratan
serta prosedur teknis pendaftaran ujian skripsi.

Setelah membahas konsep dasar dan posisi skripsi dalam kurikulum, buku
ini akan memandu pembaca menyelami prinsip-prinsip penting yang harus
dikuasai mahasiswa untuk memulai petualangannya dalam menulis skripsi.
Salah satu kunci sukses dalam penulisan skripsi adalah pemilihan topik.
Topik yang dipilih tidak hanya harus sesuai dengan minat dan keahlian
mahasiswa, tetapi juga relevan dengan bidang ilmu pengetahuan yang
sedang dikaji. Topik yang baik adalah topik yang dapat memberikan
kontribusi bagi pengembangan ilmu pengetahuan, sekaligus mampu menjawab
masalah aktual yang ada di masyarakat. Dalam buku ini, pembaca akan
diajak untuk memahami berbagai pertimbangan dalam memilih topik skripsi,
serta strategi dalam mengembangkan ide penelitian menjadi sebuah
proposal penelitian yang solid dan meyakinkan.

Selanjutnya, buku ini juga akan membahas secara sekilas mengenai teknik
pengumpulan data, baik kualitatif maupun kuantitatif, serta cara-cara
untuk menganalisis data tersebut. Mahasiswa akan diajarkan bagaimana
cara menginterpretasikan hasil penelitian dan menyajikannya dalam bentuk
narasi ilmiah yang koheren dan logis. Selain itu, aspek penting lainnya
seperti penulisan daftar pustaka, pengutipan sumber, dan penyusunan
lampiran juga akan dibahas untuk memastikan bahwa skripsi yang
dihasilkan tidak hanya berkualitas tinggi dari segi konten, tetapi juga
memenuhi standar akademik yang berlaku.

Dengan memahami isi dari buku ini, diharapkan mahasiswa dapat menavigasi
proses penulisan skripsi dengan lebih mudah dan efisien, serta pada
akhirnya dapat menghasilkan skripsi yang tidak hanya memenuhi standar
akademik, tetapi juga dapat menjadi kontribusi yang berarti bagi
pengembangan ilmu pengetahuan dan masyarakat.

\hfill\break

\textbf{Tim Penulis},

Agustus 2024




\end{document}
