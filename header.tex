% Warna judul agar mirip contoh (beige keemasan halus)
\usepackage{xcolor}
\definecolor{titleaccent}{RGB}{183,176,138} % #b7b08a

% Paket yang diperlukan
\usepackage{pdfpages}   % untuk menyisipkan cover.pdf
\usepackage{graphicx}   % untuk cover berbentuk gambar (opsional)

% Nonaktifkan title bawaan Pandoc (\maketitle) agar tidak dobel
\makeatletter
\AtBeginDocument{\let\maketitle\relax}
\makeatother

% --- Pastikan judul bab & isi berada di halaman yang sama (KOMA-Script) ---
%\KOMAoptions{titlepage=false,open=any} % redundan dengan YAML, tapi aman dipaksa di runtime

% Pangkas jeda atas/bawah judul bab agar tidak “lompat” ke halaman berikut
%\RedeclareSectionCommand[
 % beforeskip=-1.2\baselineskip, % negatif = teks langsung menyusul di halaman yang sama
  %afterskip=.8\baselineskip,
  %afterindent=false
%]{chapter}

\usepackage{afterpage}
\graphicspath{{images/}} 

\usepackage{fontspec}
\setmainfont{Georgia}
\setsansfont{Georgia}
\setmonofont{Courier New} % monospaced font alternatif, karena Georgia tidak punya varian mono
